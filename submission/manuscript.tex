\documentclass[11pt,]{article}
\usepackage{lmodern}
\usepackage{amssymb,amsmath}
\usepackage{ifxetex,ifluatex}
\usepackage{fixltx2e} % provides \textsubscript
\ifnum 0\ifxetex 1\fi\ifluatex 1\fi=0 % if pdftex
  \usepackage[T1]{fontenc}
  \usepackage[utf8]{inputenc}
\else % if luatex or xelatex
  \ifxetex
    \usepackage{mathspec}
  \else
    \usepackage{fontspec}
  \fi
  \defaultfontfeatures{Ligatures=TeX,Scale=MatchLowercase}
\fi
% use upquote if available, for straight quotes in verbatim environments
\IfFileExists{upquote.sty}{\usepackage{upquote}}{}
% use microtype if available
\IfFileExists{microtype.sty}{%
\usepackage{microtype}
\UseMicrotypeSet[protrusion]{basicmath} % disable protrusion for tt fonts
}{}
\usepackage[margin=1.0in]{geometry}
\usepackage{hyperref}
\hypersetup{unicode=true,
            pdfborder={0 0 0},
            breaklinks=true}
\urlstyle{same}  % don't use monospace font for urls
\usepackage{graphicx,grffile}
\makeatletter
\def\maxwidth{\ifdim\Gin@nat@width>\linewidth\linewidth\else\Gin@nat@width\fi}
\def\maxheight{\ifdim\Gin@nat@height>\textheight\textheight\else\Gin@nat@height\fi}
\makeatother
% Scale images if necessary, so that they will not overflow the page
% margins by default, and it is still possible to overwrite the defaults
% using explicit options in \includegraphics[width, height, ...]{}
\setkeys{Gin}{width=\maxwidth,height=\maxheight,keepaspectratio}
\IfFileExists{parskip.sty}{%
\usepackage{parskip}
}{% else
\setlength{\parindent}{0pt}
\setlength{\parskip}{6pt plus 2pt minus 1pt}
}
\setlength{\emergencystretch}{3em}  % prevent overfull lines
\providecommand{\tightlist}{%
  \setlength{\itemsep}{0pt}\setlength{\parskip}{0pt}}
\setcounter{secnumdepth}{0}
% Redefines (sub)paragraphs to behave more like sections
\ifx\paragraph\undefined\else
\let\oldparagraph\paragraph
\renewcommand{\paragraph}[1]{\oldparagraph{#1}\mbox{}}
\fi
\ifx\subparagraph\undefined\else
\let\oldsubparagraph\subparagraph
\renewcommand{\subparagraph}[1]{\oldsubparagraph{#1}\mbox{}}
\fi

%%% Use protect on footnotes to avoid problems with footnotes in titles
\let\rmarkdownfootnote\footnote%
\def\footnote{\protect\rmarkdownfootnote}

%%% Change title format to be more compact
\usepackage{titling}

% Create subtitle command for use in maketitle
\newcommand{\subtitle}[1]{
  \posttitle{
    \begin{center}\large#1\end{center}
    }
}

\setlength{\droptitle}{-2em}

  \title{}
    \pretitle{\vspace{\droptitle}}
  \posttitle{}
    \author{}
    \preauthor{}\postauthor{}
    \date{}
    \predate{}\postdate{}
  
\usepackage{helvet} % Helvetica font
\renewcommand*\familydefault{\sfdefault} % Use the sans serif version of the font
\usepackage[T1]{fontenc}

\usepackage[none]{hyphenat}

\usepackage{setspace}
\doublespacing
\setlength{\parskip}{1em}

\usepackage{lineno}

\usepackage{pdfpages}

\usepackage{amsmath}

\usepackage{mathtools}

\begin{document}

\vspace{25mm}
\begin{center}


\textbf{\huge The impact of DNA polymerase and number of rounds of amplification in PCR on 16S rRNA gene sequence data}

\vspace{50mm}



\textbf{Running title:} Quantifying the effects of PCR conditions

\vspace{25mm}

Marc A Sze${^1}$ and Patrick D Schloss${^1}$${^\dagger}$

\vspace{20mm}

$\dagger$ To whom correspondence should be addressed: \texttt{pschloss@umich.edu}

$1$ Department of Microbiology and Immunology, University of Michigan, Ann Arbor, MI

\end{center}

\newpage
\linenumbers

\hypertarget{abstract}{%
\subsection{Abstract}\label{abstract}}

PCR amplification of 16S rRNA genes is a critical, yet under appreciated
step in the generation of sequence data to describe the taxonomic
composition of microbial communities. Numerous factors in the design of
PCR can impact the sequencing error rate, the abundance of chimeric
sequences, and the degree to which the fragments in the product
represent their abundance in the original sample (i.e.~bias). We
compared the performance of high fidelity polymerases and varying number
of rounds of amplification when amplifying a mock community and human
stool samples. Although it was impossible to derive specific
recommendations, we did observe general trends. Namely, using a
polymerase with the highest possible fidelity and minimizing the number
of rounds of PCR reduced the sequencing error rate, fraction of chimeric
sequences, and bias. Evidence of bias at the sequence level was subtle
and could not be ascribed to the fragments' fraction of bases that were
guanines or cytosines. When analyzing mock community data, the amount
that the community deviated from the expected composition increased with
rounds of PCR. This bias was inconsistent for human stool samples.
Overall the results underscore the difficulty of comparing sequence data
that are generated by different PCR protocols. However, the results
indicate that the variation in human stool samples is generally larger
than that introduced by the choice of polymerase or number of rounds of
PCR.

\hypertarget{importance}{%
\subsubsection{Importance}\label{importance}}

A steep decline in sequencing costs drove an explosion in studies
characterizing microbial communities from diverse environments. Although
a significant amount of effort has gone into understanding the error
profiles of DNA sequencers, little has been done to understand the
downstream effects of the PCR amplification protocol. We quantified the
effects of the choice of polymerase and number of PCR cycles on the
quality of downstream data. We found that these choices can have a
profound impact on the way that a microbial community is represented in
the sequence data. The effects are relatively small compared to the
variation in human stool samples, however, care should be taken to use
polymerases with the highest possible fidelity and to minimize the
number of rounds of PCR. These results also underscore that it is not
possible to directly compare sequence data generated under different PCR
conditions.

\newpage

\hypertarget{introduction}{%
\subsection{Introduction}\label{introduction}}

16S rRNA gene sequencing is a powerful and widely used tool for
surveying the structure of microbial communities (1--3). This approach
has exploded in popularity with advances in sequencing throughput such
that it is now possible to characterize numerous samples with thousands
of sequences per sample. Many factors can impact how a natural community
is represented by the sequencing data including the method of acquiring
samples (4--8), storage conditions (4--6, 9--12), extraction methods
(13), amplification conditions (8, 14, 15), sequencing method (15--17),
and analytical pipeline (15, 18--20). The increased sampling depth that
is now available relative to previous Sanger sequencing-based methods is
expected to compound the impacts of an investigator's choices and the
interpretation of their results.

One step in the generation of 16S rRNA gene sequence data that has been
long known to have a significant impact on the description of microbial
communities is the choice of conditions for PCR amplification (8, 14,
15). Factors such as the choice of primers have an obvious impact on
which populations will be amplified (18, 21). However, a variety of PCR
artifacts can also impact the perception of a community including the
formation of chimeras (14, 22--24), misincorporation of nucleotides (23,
25, 26), preferential amplification of some populations over others
leading to bias (24, 27--33), and accumulation of random amplification
events leading to PCR drift (24, 27, 32, 34). Many bioinformatic tools
have been developed to identify chimeras; however, there are significant
sensitivity and specificity tradeoffs (14, 35). Laboratory-based
solutions to minimize chimera formation have also been proposed such as
minimizing the amount of template DNA in the PCR, minimizing the number
of rounds of PCR, minimizing the amount of shearing in the template DNA,
using DNA polymerases that have a proof-reading ability, and emulsion
PCR (14, 23, 36). Others have attempted to account for PCR bias using
modeling approaches (29, 37). In cases where such modeling approaches
have been successful, it has been with relatively small communities with
consistent composition (29). To minimize PCR drift, some investigators
pool technical replicate PCRs hoping to average out the drift (34).
Other factors that have been shown to impact the formation of PCR
artifacts are outside the control of an investigator including the
fraction of DNA bases that are guanines or cytosines, the variation in
the length of the targeted region across the community, the sequence of
the DNA that flanks the template, and the genetic diversity of the
community (28, 30--33). Early investigations of the factors that lead to
the formation of PCR artifacts focused on analyzing binary mixtures of
genomic DNA and 16S rRNA gene fragments to explore PCR biases and
chimera formation. Although these studies were instrumental in forcing
researchers to be cautious about the interpretation of their results, we
have a poor understanding of how these factors affect the formation of
PCR artifacts in more complex communities.

The influence that the choice of DNA polymerase has on the formation of
PCR artifacts has not been well studied. There has been recent interest
in how the choice of the hypervariable region and data analysis
pipelines impact the sequencing error rate (15, 18--20); however, these
studies use the same DNA polymerase in the PCR step and implicitly
assume that the rate of nucleotide misincorporation from PCR are
significantly smaller than those from the sequencing phase. There has
been more limited interest in the impact that DNA polymerase choice has
on the formation of chimeras (23, 38). A recent study found differences
in the number of OTUs and chimeras between normal and high fidelity DNA
polymerases (38). The authors of the study reduced the difference
between two polymerases by optimizing the annealing and extension steps
within the PCR protocol (38). Yet this optimization was specific for the
community they were analyzing (i.e.~captive and semi-captive red-shanked
doucs) and assumed that if the two polymerases generate the same
community structure that the community structure was correct. In fact,
the community structure generated by both methods was not free of
artifacts, but likely had the same artifacts. A challenge in these types
of experiments is having \emph{a priori} knowledge of the true community
representation. A mock community with known composition allows
researchers to quantify the sequencing error rate, fraction of chimeras,
and bias (19); however, mock communities have a limited phylogenetic
diversity relative to natural communities. Natural communities, in
contrast, have an unknown community composition making absolute
measurements impossible. They can be used to validate results from mock
communities and to understand the relative impacts of artifacts on the
ability to differentiate biological and methodological sources of
variation. Given the large number of DNA polymerases available to
researchers, it is unlikely that a specific recommendation is possible.
Rather, the development of general best practices and understanding the
impact of PCR artifacts on an analysis are needed.

This study investigated the impact of choice of high fidelity DNA
polymerase and the number of rounds of amplification on the formation of
PCR artifacts using a mock community and human stool samples. It was
hypothesized that additional rounds of PCR would exacerbate the number
of artifacts. We tested (i) the effect of the polymerase on the error
rate of the bases represented in the final sequences, (ii) the fraction
of sequences that appeared to be chimeras and the ability to detect
those chimeras, (iii) the bias of preferentially amplifying one fragment
over another in a mixed pool of templates, and (iv) inter-sample
variation in community structure of samples amplified with the same
polymerase across the amplification process. To characterize these
factors we sequenced a mock community of 8 organisms with known
sequences and community structure and human fecal samples with unknown
sequences and community structures. We sequenced the V4 region of the
16S rRNA genes from a mock community by generating paired 250 nt reads
on the Illumina MiSeq platform. This region and sequencing approach was
used because it has been shown to result in a relatively low sequencing
error rate and is a widely used protocol (18). To better understand the
impact of DNA polymerase choice on PCR artifacts, we selected five high
fidelity DNA polymerases and amplified the communities using 20, 25, 30,
and 35 rounds of amplification. Collectively, our results suggest that
the number of rounds and to a lesser extent the choice of DNA polymerase
used in PCR impact the sequence data. The effects are consistent and are
smaller than the biological differences between individuals.

\newpage

\hypertarget{results}{%
\subsection{Results}\label{results}}

\textbf{\emph{Sequencing errors vary by the number of cycles and the DNA
polymerase used in PCR.}} The presence of sequence errors can confound
the ability to accurately classify 16S rRNA gene sequences and group
sequences into operational taxonomic units (OTUs). More importantly,
sequencing errors themselves can alter the representation of the
community. Therefore, it is important to minimize the number of
sequencing errors. Using a widely-used approach that generates the
lowest reported error rate, we quantified the error rate by sequencing
the V4 region of the 16S rRNA genes from an 8 member mock community. We
also removed any contigs that were at least three bases more similar to
a chimera of two references than to a single reference sequence (18, 19,
39). Regardless of the polymerase, the error rate increased with the
number of rounds of amplification (Figure 1). Using 30 rounds of PCR is
a common approach across diverse types of samples. Among the data
generated using 30 rounds of PCR the Accuprime polymerase had the
highest error rate (i.e.~0.124\%) followed by the Platinum
(i.e.~0.094\%), Phusion (i.e.~0.064\%), KAPA (i.e.~0.062\%), and Q5
(i.e.~0.060\%) polymerases (Figure 1). When we applied a pre-clustering
denoising step, which merged the counts of reads within 2 nt of a more
abundant sequence (19), the error rates dropped considerably such that
the Platinum polymerase had the highest error rate (i.e.~0.014\%)
followed by the Accuprime (i.e.~0.012\%), Q5 (i.e.~0.0053\%), Phusion
(i.e.~0.0049\%), and KAPA (i.e.~0.0049\%) polymerases (Figure 1).
Although specific recommendations are difficult to make because the
phylogenetic diversity of the initial DNA template is likely to have an
impact on the results, it is clear that using as few PCR cycles as
necessary and a polymerase with the lowest possible error rate is a good
guide to minimizing the impact of polymerase on the error rate.

\textbf{\emph{The fraction of sequences identified as being chimeric
varies by the number of cycles and the DNA polymerase used in PCR.}}
Chimeric PCR products can significantly confound downstream analyses.
Although numerous bioinformatic tools exist to identify and remove
chimeric sequences with high specificity, their sensitivity is
relatively low and can be reduced by the presence of sequencing errors
(14, 35). Because the true sequences of the organisms in the mock
community were known, we generated all possible chimeras between pairs
of V4 16S rRNA gene fragments and used these possible chimeric sequences
to screen the sequences generated under the different PCR conditions to
detect chimeras. The number of chimeras increased with rounds of
amplification (Figure 2A). Interestingly, the fraction of chimeric
sequences from the mock community varied by the type of polymerase used.
After 30 rounds of PCR, the Platinum polymerase had the highest chimera
rate (i.e.~18.2\%) followed by the Q5 (i.e.~8.1\%), Phusion
(i.e.~7.5\%), KAPA (i.e.~2.3\%), and Accuprime (i.e.~0.9\%) polymerases.
To explore the characteristics of the chimeras further, we analyzed
those chimeras formed after 35 cycles. Because of the uneven number of
chimeras generated across the five polymerases, we subsampled the
frequency of the chimeras to have the same number of chimeras per
polymerase the Q5, Phusion, Accuprime, and Platinum polymerases; the
chimeric sequence yield with the KAPA polymerase was significantly lower
than the other polymerases and was omitted from our initial comparison.
As has been shown previously (14), chimera formation was not random.
Among the chimeras that were generated in mock community samples, 4.4\%
of the chimeras were found across all four polymerases. These chimeras
represented between 67.6 and 74.5\% of the chimeras generated with each
polymerase; they represented 40.4\% of the chimeric sequences generated
using the KAPA polymerase. These results indicate that the mechanisms
leading to the formation of chimeras are largely independent of the
properties of the polymerase, but are more likely due to the properties
of the sequences.

Because our chimera screening procedure could only be applied to mock
communities, we used the UCHIME algorithm to model the chimera screening
approach that is used in most sequence curation pipelines. By comparing
the output of UCHIME to our approach of screening for chimeras using all
possible chimeras generated from the mock community sequences, we were
able to calculate the UCHIME's sensitivity and specificity (Figure 2A).
The specificity for all polymerases was above 95.4\% and showed a weak
association with the number of cycles used (Figure 2A). There was
considerable inter-polymerase and inter-round of amplification variation
in the sensitivity of UCHIME to detect the chimeras from the mock
community. This suggested that the residual error rate after
pre-clustering the sequence data did not compromise the sensitivity of
UCHIME to detect chimeras. The sensitivity of UCHIME varied between 50
and 87.0\% when at least 25 cycles were used. The generalizability of
these results is limited because we used a single mock community with
limited genetic diversity. Although we did not know the true chimera
rate for our four human stool samples, we were able to calculate the
fraction of sequences that UCHIME identified as being chimeric (Figure
2B). These results followed those from the mock communities: additional
rounds of amplification significantly increased the rate of chimeras and
there was variation between the polymerases that we used. Although it
was not possible to identify the features of a polymerase that resulted
in higher rates of chimeras, it is clear that using the smallest number
of PCR cycles possible will minimize the impact of chimeras.

\textbf{\emph{At the sequence level, PCR amplification bias is subtle.}}
Since researchers began using PCR to amplify 16S rRNA gene fragments
there has been concern that amplifying fragments from a mixed template
pool could lead to a biased representation in the pool of products and
would confound downstream analyses (24, 27--33). The mock community was
generated by mixing equal amounts of genomic DNA from 8 bacteria
resulting in uneven representation of the \emph{rrn} operons across the
bacteria as each bacterium had a different genome size and varied in the
number of operons in its genome. The vendor of the mock community
subjects each lot of genomic DNA to shotgun sequencing to more
accurately quantify the actual abundance of each organism in the
community. It should be noted that this approach to quantifying
abundance is also not without its own biases, but does provide an
alternative approach to characterizing the structure of the mock
community. We compared the vendor reported relative abundance of the 16S
rRNA genes from each bacterium in the mock community to the data we
generated across rounds of amplification and polymerase (Figure 3).
Interestingly, for some bacteria, their representation became less
biased with additional rounds of PCR (e.g.~\emph{L. fermentum}), while
others became more biased (e.g.~\emph{E. faecalis}), and others had
little change (e.g.~\emph{B. subtilis}). Contrary to prior reports (28),
the percentage of bases in the V4 region that were guanines or cytosines
was not predictive of the amount of bias. Across the strains there was
no variation in the length of their V4 regions and they each had the
same sequence in the region that the primers annealed. One of the
bacteria represented in the mock community, \emph{S. enterica}, had 6
identical copies of the V4 region and 1 copy that differed from those by
one nucleotide. The dominant copy had a thymidine and the rare copy had
a guanine. We used the sequence data to calculate the ratio of the
dominant to rare variants from \emph{S. enterica} expecting a ratio near
6 (Figure S1). The Accuprime, Phusion, Platinum, and Q5 polymerases
converged to a ratio of 5.4; however, the ratio for the KAPA polymerase
was above 6 for all rounds of PCR (6.1-7.4) and the ratio for Q5 was
below 6 for all rounds of PCR (5.3-5.5). Given the subtle nature of the
variation in the relative abundances of each 16S rRNA gene fragment, it
was not possible to create generalizable rules that would explain the
bias.

\textbf{\emph{At the community level, the effects of PCR amplification
bias grow with additional rounds of PCR.}} Because the variation in bias
between polymerases and across rounds of PCR could be artificially
inflated due to sequencing errors and chimeras, we analyzed the alpha
and beta diversity of the mock community data at different phases of the
sequence curation pipeline (Figure 4). First, we removed the chimeras
from the mock community data as described above and mapped the
individual reads to the OTUs that the 16S rRNA gene fragments would
cluster into if there were no sequencing errors. This gave us a
community distribution that reflected the distribution following PCR
without any artifacts (Figure 4A; ``No errors or chimeras''). Although
the richness did not change, the Shannon diversity increased with the
number of rounds of PCR for all polymerases except the KAPA polymerase,
for which the diversity decreased. These data suggest that PCR had the
effect of making the community distribution more even than it was
originally, except for the data generated using the KAPA polymerase
where the evenness decreased. Next, we used the observed sequence
errors, but removed chimeras by comparing sequences to all possible
chimeras between mock community sequences, and clustered the reads to
OTUs (Figure 4A; ``Residual errors, complete chimera removal''). The
richness and diversity metrics trended higher with higher error rates
and number of rounds of PCR. Finally, we used the observed sequence data
and the UCHIME algorithm to identify chimeras (Figure 4A; ``Residual
errors, chimera removal with VSEARCH''). Again, the richness and
diversity metrics trended higher with higher error rates and number of
rounds of PCR. These comparisons demonstrated that although the bias at
the sequence level was subtle, PCR introduces bias at the community
level that is exacerbated by errors and chimeras when sequences are
clustered into OTUs. When we measured the Bray-Curtis distance between
the communities observed after 25 rounds of amplification and those at
30 and 35, distances between 25 and 35 rounds were higher than between
25 and 30 rounds for each of the polymerases by an average of 0.022
units (Figure 4B). The Platinum polymerase varied the most across rounds
of amplification (25 vs 30 rounds: 0.13; 25 vs 35 rounds: 0.15). For any
number of cycles, the median Bray-Curtis distance between polymerases
ranged between 0.074 and 0.11. Although the distances between samples
were small, the ordination of these distances showed a clear change in
community structure with increasing rounds of PCR (Figure 4C). This
observation was supported by our statistical analysis, which revealed
that the effect of the number of rounds of PCR
(R\textsuperscript{2}=0.21, P\textless{}0.001) was comparable to the
choice of polymerase (R\textsuperscript{2}=0.20, P\textless{}0.001).
These results demonstrate that subtle differences in relative abundances
can have an impact on overall community structure. This variation
underscores the importance of only comparing sequence data that have
been generated using the same PCR conditions.

\textbf{\emph{The choice of polymerase or the number of rounds of
amplification have little impact on the relative interpretation of
community-wide metrics of diversity.}} We expected that the biases that
we observed at the population and community levels using mock community
data would be small relative to the expected differences between
biological samples. To study this further, we calculated alpha and
beta-diversity metrics using the human stool samples for each of the
polymerases and rounds of amplification. We calculated the number of
observed OTUs and Shannon diversity for each condition and stool sample
(Figure 5A). Although there were clear differences between PCR
conditions, the relative ordering of the stool samples did not
meaningfully vary across conditions. When we characterized the variation
between rounds of amplification using human stool samples, the distance
between the 25 and 30 rounds and 25 and 35 rounds varied considerably
between samples and polymerases (Figure 5B). In general the inter-round
variation was lowest for the data generated using the KAPA and Accuprime
polymerases. The data generated using the Platinum polymerase was
consistent across rounds, but overall, it was more biased than the other
polymerases. Considering the average distance across the four samples
varied between 0.39 and 0.56, regardless of the polymerases and number
of rounds of amplification, any bias due to amplification is unlikely to
obscure community-wide differences between samples. In support of this
was our principle coordinates analysis of the Bray-Curtis distances,
which revealed distinct clusters by stool sample (Figure 5C). Within
each cluster there were no obvious patterns related to the polymerase or
number of rounds of PCR. Our statistical analysis revealed statistically
significant differences in the community structures with the stool
sample explaining the most variation (R\textsuperscript{2}=0.79,
P\textless{}0.001), followed by the number of rounds of PCR
(R\textsuperscript{2}=0.044, P\textless{}0.001) and the choice of
polymerase (R\textsuperscript{2}=0.033, P\textless{}0.001). Together,
these results indicate that for a coarse analysis of communities, the
choice of number of rounds of amplification or polymerase are not
important, but that they must be consistent across samples. It is
difficult to develop a specific recommendation based on the level of
bias across rounds of PCR or polymerases; however, the general
suggestion is to use as few rounds of amplification as possible.

\textbf{\emph{There is little evidence of a relationship between
polymerase or number of rounds of amplification on PCR drift.}} There
have been concerns that the same template DNA subjected to the same PCR
conditions could result in different representations of communities
because of random drift over the course of PCR. To test this, we
determined the average Bray-Curtis distance between replicate reactions
using the same polymerase and number of rounds of amplification (Figure
6). Using the mock community data there were no obvious trends. The
average Bray-Curtis distance within a set of conditions varied by 0.062
to 0.11 units. Although we did not generate technical replicates of each
of the stool samples, the inter-sample variation for each set of
conditions was consistent and varied between 0.50 and 0.56 units. These
data suggest that amplicon sequencing is robust to random variation in
amplification and that any differences are likely to be smaller than
what is considered biologically relevant.

\newpage

\hypertarget{discussion}{%
\subsection{Discussion}\label{discussion}}

Our results suggest that the number of rounds of PCR and to a lesser
degree the choice of DNA polymerase impact the analysis of 16S rRNA gene
sequence data from bacterial communities. Although it was not possible
to make direct connections between PCR conditions and specific sources
of bias, we were able to identify general recommendations that reduce
the amount of error, chimera formation, and bias. Researchers should
strive to minimize the number of rounds of PCR and should use a high
fidelity polymerase. Although specific PCR conditions impact the precise
interpretation of the data, the effects were consistent and were smaller
than the biological differences between the samples we tested. Based on
these observations, amplicons must be generated by consistent protocols
to yield meaningful comparisons. When comparing across studies, values
like richness, diversity, and relative abundances must be made in
relative and not absolute terms. Furthermore, care must be taken to not
directly compare or pool samples from different studies. Instead, it is
important to statistically model the study-based variation as has been
done in recent meta-analyses that compared relative effect sizes or
pooled data using a mixed effects statistical model (40, 41).

The observed sequencing error rates and alpha diversity metrics followed
the manufacturers' measurements of their polymerases' fidelity (Figure
1). Accuprime and Platinum have fidelity that are approximately 10-times
higher than that of Taq whereas the fidelity of Phusion, Q5, and KAPA
are more than 100 times higher. Among these polymerases, the KAPA
polymerase consistently resulted in a lower error rate, lower chimera
rate, and lower bias across rounds of PCR for the mock community
samples. Furthermore, among the human samples, the KAPA polymerase
consistently had the lowest detected chimera rate and inter-cycle bias.
These benefits were most accentuated at 35 cycles. However, in our
experience and despite efforts to optimize the yield with the KAPA
polymerase, the reactions typically had a high proportion of
primer-dimer products and low yield of correctly-sized products.
Although the error rate for with the Accuprime polymerase was not as low
as that with KAPA, we consider it to be an acceptable alternative.
Considering polymerase development is an active area of commercial
development with potential new polymerases becoming available, it is
important for researchers to understand how changing the polymerase
impacts downstream analyses for their type of samples.

Over the past 20 years, a large literature has attempted to document
various PCR biases and underscored the fact that data based on
amplification of DNA from a mixed community are not a true
representation of the actual community. In addition to obvious biases
imposed by primer selection, other factors inherent in PCR can influence
the representation of communities. Factors that can lead to preferential
amplification of one fragment over another have included guanine and
cytosine composition, length, flanking DNA composition, amount of DNA
shearing, and number of rounds of PCR (24, 27--33). These factors may
become exacerbated if PCR is performed on multiple samples that vary in
their concentration (42). In addition, environmental and reagent
contaminants can also have a significant impact on the analysis of low
biomass samples (43). Less well understood is the effect of phylogenetic
diversity on bias and chimera formation. Communities with low
phylogenetic diversity may be more prone to chimera formation since
chimeras are more likely to form among closely related sequences (14,
35). The interaction of these various influences on PCR artifacts are
complex and difficult to tease apart. Minimizing the level of DNA
shearing, controlling for template concentration across samples, and
using the fewest number of rounds of PCR with a polymerase that has the
highest possible fidelity are strategies that can be employed to
minimize the formation of chimeras. Although care should always be taken
when choosing a polymerase for 16S rRNA gene sequencing, our
observations show that variation among polymerases is smaller than the
actual biological variation in fecal communities between individuals.

Even with these strategies it is impossible to remove all PCR artifacts.
Beyond the imperfections of the best polymerases, sometimes difficult to
lyse organisms require stringent lysis steps and low biomass samples
require additional rounds of PCR. A host of bioinformatics tools are
available for removing residual sequencing errors (18, 44--46). Other
tools are available for removing chimeras (14, 35) where there is a
trade off between the sensitivity of detecting chimeras and the
specificity of correctly calling a sequence a chimera. In recent years,
parameters for these algorithms have been changed to increase their
sensitivity with little evaluation of the effects on the specificity of
the algorithms (44, 46). Others recommend removing any read that has an
abundance below a specified threshold as a tool to remove PCR and
sequencing artifacts (e.g.~removing all sequences that only appear once)
(20, 44--46). This method must be approached with caution as such
approaches are likely to introduce a different bias of the community
representation and ignore the fact, as we showed, that artifacts may be
quite abundant and reproducible. Ultimately, researchers must test their
hypotheses with multiple methods to validate the claims they reach with
any one method (47). All methods have biases and limitations and we must
use complementary methods to obtain robust results.

\newpage

\hypertarget{materials-methods}{%
\subsection{Materials \& Methods}\label{materials-methods}}

\textbf{\emph{Mock community.}} The ZymoBIOMICS\textsuperscript{TM}
Microbial Community DNA Standard (Zymo, CA, USA) was used for mock
communities and the bacterial component was made up of \emph{Pseudomonas
aeruginosa}, \emph{Escherichia coli}, \emph{Salmonella enterica},
\emph{Lactobacillus fermentum}, \emph{Enterococcus faecalis},
\emph{Staphylococcus aureus}, \emph{Listeria monocytogenes}, and
\emph{Bacillus subtilis} at equal genomic DNA abundance
(\url{https://web.archive.org/web/20171217151108/http://www.zymoresearch.com:80/microbiomics/microbial-standards/zymobiomics-microbial-community-standards}).
The actual relative abundance for each bacterium was obtained from
Zymo's certificate of analysis for the lot (Lot: ZRC187325), which they
determined using shotgun metagenomic sequencing
(\url{https://github.com/SchlossLab/Sze_PCRSeqEffects_mSphere_2019/data/references/ZRC187325.pdf}).

\textbf{\emph{Human samples.}} Fecal samples were obtained from 4
individuals who were part of an earlier study (48). These samples were
collected using a protocol approved by the University of Michigan
Institutional Review Board. For this study, the samples were
de-identified. DNA was extracted from the fecal samples using the
MOBIO\textsuperscript{TM} PowerMag Microbiome RNA/DNA extraction kit
(now Qiagen, MD, USA).

\textbf{\emph{PCR protocol.}} Five high fidelity DNA polymerases were
tested including AccuPrime\textsuperscript{TM} (ThermoFisher, MA, USA),
KAPA HIFI (Roche, IN, USA), Phusion (New England Biolabs, MA, USA),
Platinum (ThermoFisher, MA, USA), and Q5 (New England Biolabs, MA, USA).
Manufacturer recommendations were followed except for the annealing and
extension times, which were selected based on previously published
protocols (18, 38). Primers targeting the V4 region of the 16S rRNA gene
were used with modifications to generate MiSeq amplicon libraries (18)
(\url{https://github.com/SchlossLab/MiSeq_WetLab_SOP/}). The 16S rRNA
gene targeting regions of the primers annealed to \emph{E. coli}
positions 515 to 533 (GTGCCAGCMGCCGCGGTAA) and 787 to 806
(GGACTACHVGGGTWTCTAAT). The number of rounds of PCR used for each sample
and polymerase started at 15 and increased by 5 rounds up to 35 cycles.
Insufficient PCR product was generated using 15 rounds and has not been
included in our analysis.

\textbf{\emph{Library generation and sequencing.}} Each PCR condition
(i.e.~combination of polymerase and number of rounds of PCR) were
replicated four times for the mock community and one time for each fecal
sample. Libraries were generated as previously described (18)
(\url{https://github.com/SchlossLab/MiSeq_WetLab_SOP/}). The libraries
were sequenced using the Illumina MiSeq sequencing platform to generate
paired 250-nt reads.

\textbf{\emph{Sequence processing.}} The mothur software program (v
1.41) was used for all sequence processing steps (49). The protocol has
been previously published (18)
(\url{https://www.mothur.org/wiki/MiSeq_SOP}). Briefly, paired reads
were assembled using mothur's make.contigs command to correct errors
introduced by sequencing (18). Any assembled contigs that contained an
ambiguous base call, mapped to the incorrect region of the 16S rRNA
gene, or appeared to be a contaminant were removed from subsequent
analyses. Sequences were further denoised using mothur's pre.cluster
command to merge the counts of sequences that were within 2 nt of a more
abundant sequence. The VSEARCH implementation of UCHIME was used to
screen for chimeras (35, 50). At various stages in the sequence
processing pipeline for the mock community data, the mothur seq.error
command was used to quantify the sequencing error rate as well as the
true chimera rate. This command uses the true sequences from the mock
community to generate all possible chimeras and removes any contigs that
were at least three bases more similar to a chimera than to a reference
sequence. The command then counts the number of substitutions,
insertions, and deletions in the contig relative to the reference
sequence and reports the error rate without the inclusion of chimeric
sequences (19). UCHIME's sensitivity was calculated as the percentage of
true chimeras that were detected as chimeras when using UCHIME. Its
specificity was calculated as the percentage of non-chimeric sequences
that were detected as being non-chimeric by UCHIME. The reference
sequences and \emph{rrn} operon copy number for each bacterium were
obtained from the ZymoBIOMICS\textsuperscript{TM} Microbial Community
DNA Standard protocol
(\url{https://web.archive.org/web/20181221151905/https://www.zymoresearch.com/media/amasty/amfile/attach/_D6305_D6306_ZymoBIOMICS_Microbial_Community_DNA_Standard_v1.1.3.pdf}).
Sequences were assigned to operational taxonomic units (OTUs) at a
threshold of 3\% dissimilarity using the OptiClust algorithm (51). To
adjust for unequal sequencing when measuring alpha and beta diversity,
all samples were rarefied for downstream analysis. The Good's coverage
for the samples was routinely greater than 95\%.

\textbf{\emph{Statistical analysis.}} All analysis was done with the R
(v 3.5.1) software package (52). Data transformation and graphing were
completed using the tidyverse package (v 1.2.1). The distance matrix
data was analyzed using the adonis function within the vegan package (v
2.5.4).

\textbf{\emph{Reproducible methods.}} The data analysis code for this
study can be found at
\url{https://github.com/SchlossLab/Sze_PCRSeqEffects_mSphere_2019}. The
raw sequences are available at the SRA (Accession SRP132931).

\hypertarget{acknowledgements}{%
\subsection{Acknowledgements}\label{acknowledgements}}

We appreciate the willingness of the donors to provide stool samples. We
also thank Judy Opp and April Cockburn for their assistance in
sequencing the samples as part of the Microbiome Core Facility at the
University of Michigan. Additional thanks to members of the Schloss lab
and Dr.~Marcy Balunas for reading earlier drafts of the manuscript and
providing helpful critiques. Support for MAS came from the Canadian
Institute of Health Research and NIH grant UL1TR002240 and support for
PDS came from NIH grants P30DK034933, R01CA215574, and U19AI09087.

\newpage

\hypertarget{references}{%
\subsection{References}\label{references}}

\hypertarget{refs}{}
\leavevmode\hypertarget{ref-Gilbert2018}{}%
1. \textbf{Gilbert JA}, \textbf{Jansson JK}, \textbf{Knight R}. 2018.
Earth microbiome project and global systems biology. mSystems
\textbf{3}.
doi:\href{https://doi.org/10.1128/msystems.00217-17}{10.1128/msystems.00217-17}.

\leavevmode\hypertarget{ref-HMP2012}{}%
2. \textbf{Human Microbiome Consortium}. 2012. Structure, function and
diversity of the healthy human microbiome. Nature \textbf{486}:207--214.
doi:\href{https://doi.org/10.1038/nature11234}{10.1038/nature11234}.

\leavevmode\hypertarget{ref-Schloss2016}{}%
3. \textbf{Schloss PD}, \textbf{Girard RA}, \textbf{Martin T},
\textbf{Edwards J}, \textbf{Thrash JC}. 2016. Status of the archaeal and
bacterial census: An update. mBio \textbf{7}.
doi:\href{https://doi.org/10.1128/mbio.00201-16}{10.1128/mbio.00201-16}.

\leavevmode\hypertarget{ref-Luo2016}{}%
4. \textbf{Luo T}, \textbf{Srinivasan U}, \textbf{Ramadugu K},
\textbf{Shedden KA}, \textbf{Neiswanger K}, \textbf{Trumble E},
\textbf{Li JJ}, \textbf{McNeil DW}, \textbf{Crout RJ}, \textbf{Weyant
RJ}, \textbf{Marazita ML}, \textbf{Foxman B}. 2016. Effects of specimen
collection methodologies and storage conditions on the short-term
stability of oral microbiome taxonomy. Applied and Environmental
Microbiology \textbf{82}:5519--5529.
doi:\href{https://doi.org/10.1128/aem.01132-16}{10.1128/aem.01132-16}.

\leavevmode\hypertarget{ref-Bassis2017}{}%
5. \textbf{Bassis CM}, \textbf{Nicholas M. Moore}, \textbf{Lolans K},
\textbf{Seekatz AM}, \textbf{Weinstein RA}, \textbf{Young VB},
\textbf{Hayden MK}. 2017. Comparison of stool versus rectal swab samples
and storage conditions on bacterial community profiles. BMC Microbiology
\textbf{17}.
doi:\href{https://doi.org/10.1186/s12866-017-0983-9}{10.1186/s12866-017-0983-9}.

\leavevmode\hypertarget{ref-Gorzelak2015}{}%
6. \textbf{Gorzelak MA}, \textbf{Gill SK}, \textbf{Tasnim N},
\textbf{Ahmadi-Vand Z}, \textbf{Jay M}, \textbf{Gibson DL}. 2015.
Methods for improving human gut microbiome data by reducing variability
through sample processing and storage of stool. PLOS ONE
\textbf{10}:e0134802.
doi:\href{https://doi.org/10.1371/journal.pone.0134802}{10.1371/journal.pone.0134802}.

\leavevmode\hypertarget{ref-Dominianni2014}{}%
7. \textbf{Dominianni C}, \textbf{Wu J}, \textbf{Hayes RB}, \textbf{Ahn
J}. 2014. Comparison of methods for fecal microbiome biospecimen
collection. BMC Microbiology \textbf{14}:103.
doi:\href{https://doi.org/10.1186/1471-2180-14-103}{10.1186/1471-2180-14-103}.

\leavevmode\hypertarget{ref-BautistadelosSantos2016}{}%
8. \textbf{Santos QMB-d los}, \textbf{Schroeder JL}, \textbf{Blakemore
O}, \textbf{Moses J}, \textbf{Haffey M}, \textbf{Sloan W}, \textbf{Pinto
AJ}. 2016. The impact of sampling, PCR, and sequencing replication on
discerning changes in drinking water bacterial community over diurnal
time-scales. Water Research \textbf{90}:216--224.
doi:\href{https://doi.org/10.1016/j.watres.2015.12.010}{10.1016/j.watres.2015.12.010}.

\leavevmode\hypertarget{ref-Sinha2015}{}%
9. \textbf{Sinha R}, \textbf{Chen J}, \textbf{Amir A}, \textbf{Vogtmann
E}, \textbf{Shi J}, \textbf{Inman KS}, \textbf{Flores R},
\textbf{Sampson J}, \textbf{Knight R}, \textbf{Chia N}. 2015. Collecting
fecal samples for microbiome analyses in epidemiology studies. Cancer
Epidemiology Biomarkers \& Prevention \textbf{25}:407--416.
doi:\href{https://doi.org/10.1158/1055-9965.epi-15-0951}{10.1158/1055-9965.epi-15-0951}.

\leavevmode\hypertarget{ref-Amir2017b}{}%
10. \textbf{Amir A}, \textbf{McDonald D}, \textbf{Navas-Molina JA},
\textbf{Debelius J}, \textbf{Morton JT}, \textbf{Hyde E},
\textbf{Robbins-Pianka A}, \textbf{Knight R}. 2017. Correcting for
microbial blooms in fecal samples during room-temperature shipping.
mSystems \textbf{2}:e00199--16.
doi:\href{https://doi.org/10.1128/msystems.00199-16}{10.1128/msystems.00199-16}.

\leavevmode\hypertarget{ref-Lauber2010}{}%
11. \textbf{Lauber CL}, \textbf{Zhou N}, \textbf{Gordon JI},
\textbf{Knight R}, \textbf{Fierer N}. 2010. Effect of storage conditions
on the assessment of bacterial community structure in soil and
human-associated samples. FEMS Microbiology Letters \textbf{307}:80--86.
doi:\href{https://doi.org/10.1111/j.1574-6968.2010.01965.x}{10.1111/j.1574-6968.2010.01965.x}.

\leavevmode\hypertarget{ref-Song2016}{}%
12. \textbf{Song SJ}, \textbf{Amir A}, \textbf{Metcalf JL},
\textbf{Amato KR}, \textbf{Xu ZZ}, \textbf{Humphrey G}, \textbf{Knight
R}. 2016. Preservation methods differ in fecal microbiome stability,
affecting suitability for field studies. mSystems \textbf{1}:e00021--16.
doi:\href{https://doi.org/10.1128/msystems.00021-16}{10.1128/msystems.00021-16}.

\leavevmode\hypertarget{ref-Costea2017}{}%
13. \textbf{Costea PI}, \textbf{Zeller G}, \textbf{Sunagawa S},
\textbf{Pelletier E}, \textbf{Alberti A}, \textbf{Levenez F},
\textbf{Tramontano M}, \textbf{Driessen M}, \textbf{Hercog R},
\textbf{Jung F-E}, \textbf{Kultima JR}, \textbf{Hayward MR},
\textbf{Coelho LP}, \textbf{Allen-Vercoe E}, \textbf{Bertrand L},
\textbf{Blaut M}, \textbf{Brown JRM}, \textbf{Carton T},
\textbf{Cools-Portier S}, \textbf{Daigneault M}, \textbf{Derrien M},
\textbf{Druesne A}, \textbf{Vos WM de}, \textbf{Finlay BB},
\textbf{Flint HJ}, \textbf{Guarner F}, \textbf{Hattori M},
\textbf{Heilig H}, \textbf{Luna RA}, \textbf{Hylckama Vlieg J van},
\textbf{Junick J}, \textbf{Klymiuk I}, \textbf{Langella P},
\textbf{Chatelier EL}, \textbf{Mai V}, \textbf{Manichanh C},
\textbf{Martin JC}, \textbf{Mery C}, \textbf{Morita H}, \textbf{O'Toole
PW}, \textbf{Orvain C}, \textbf{Patil KR}, \textbf{Penders J},
\textbf{Persson S}, \textbf{Pons N}, \textbf{Popova M}, \textbf{Salonen
A}, \textbf{Saulnier D}, \textbf{Scott KP}, \textbf{Singh B},
\textbf{Slezak K}, \textbf{Veiga P}, \textbf{Versalovic J}, \textbf{Zhao
L}, \textbf{Zoetendal EG}, \textbf{Ehrlich SD}, \textbf{Dore J},
\textbf{Bork P}. 2017. Towards standards for human fecal sample
processing in metagenomic studies. Nature Biotechnology.
doi:\href{https://doi.org/10.1038/nbt.3960}{10.1038/nbt.3960}.

\leavevmode\hypertarget{ref-Haas2011}{}%
14. \textbf{Haas BJ}, \textbf{Gevers D}, \textbf{Earl AM},
\textbf{Feldgarden M}, \textbf{Ward DV}, \textbf{Giannoukos G},
\textbf{Ciulla D}, \textbf{Tabbaa D}, \textbf{Highlander SK},
\textbf{Sodergren E}, \textbf{Methe B}, \textbf{DeSantis TZ},
\textbf{Petrosino JF}, \textbf{Knight R}, \textbf{and BWB}. 2011.
Chimeric 16S rRNA sequence formation and detection in sanger and
454-pyrosequenced PCR amplicons. Genome Research \textbf{21}:494--504.
doi:\href{https://doi.org/10.1101/gr.112730.110}{10.1101/gr.112730.110}.

\leavevmode\hypertarget{ref-Sinha2017}{}%
15. \textbf{Sinha R}, \textbf{Abu-Ali G}, \textbf{Vogtmann E},
\textbf{Fodor AA}, \textbf{Ren B}, \textbf{Amir A}, \textbf{Schwager E},
\textbf{Crabtree J}, \textbf{Ma S}, \textbf{Abnet CC}, \textbf{Knight
R}, \textbf{White O}, \textbf{Huttenhower C}. 2017. Assessment of
variation in microbial community amplicon sequencing by the microbiome
quality control (MBQC) project consortium. Nature Biotechnology.
doi:\href{https://doi.org/10.1038/nbt.3981}{10.1038/nbt.3981}.

\leavevmode\hypertarget{ref-Meisel2016}{}%
16. \textbf{Meisel JS}, \textbf{Hannigan GD}, \textbf{Tyldsley AS},
\textbf{SanMiguel AJ}, \textbf{Hodkinson BP}, \textbf{Zheng Q},
\textbf{Grice EA}. 2016. Skin microbiome surveys are strongly influenced
by experimental design. Journal of Investigative Dermatology
\textbf{136}:947--956.
doi:\href{https://doi.org/10.1016/j.jid.2016.01.016}{10.1016/j.jid.2016.01.016}.

\leavevmode\hypertarget{ref-Caporaso2010}{}%
17. \textbf{Caporaso JG}, \textbf{Lauber CL}, \textbf{Walters WA},
\textbf{Berg-Lyons D}, \textbf{Lozupone CA}, \textbf{Turnbaugh PJ},
\textbf{Fierer N}, \textbf{Knight R}. 2010. Global patterns of 16S rRNA
diversity at a depth of millions of sequences per sample. Proceedings of
the National Academy of Sciences \textbf{108}:4516--4522.
doi:\href{https://doi.org/10.1073/pnas.1000080107}{10.1073/pnas.1000080107}.

\leavevmode\hypertarget{ref-Kozich2013}{}%
18. \textbf{Kozich JJ}, \textbf{Westcott SL}, \textbf{Baxter NT},
\textbf{Highlander SK}, \textbf{Schloss PD}. 2013. Development of a
dual-index sequencing strategy and curation pipeline for analyzing
amplicon sequence data on the MiSeq illumina sequencing platform.
Applied and Environmental Microbiology \textbf{79}:5112--5120.
doi:\href{https://doi.org/10.1128/aem.01043-13}{10.1128/aem.01043-13}.

\leavevmode\hypertarget{ref-Schloss2011}{}%
19. \textbf{Schloss PD}, \textbf{Gevers D}, \textbf{Westcott SL}. 2011.
Reducing the effects of PCR amplification and sequencing artifacts on
16S rRNA-based studies. PLoS ONE \textbf{6}:e27310.
doi:\href{https://doi.org/10.1371/journal.pone.0027310}{10.1371/journal.pone.0027310}.

\leavevmode\hypertarget{ref-Bokulich2012}{}%
20. \textbf{Bokulich NA}, \textbf{Subramanian S}, \textbf{Faith JJ},
\textbf{Gevers D}, \textbf{Gordon JI}, \textbf{Knight R}, \textbf{Mills
DA}, \textbf{Caporaso JG}. 2012. Quality-filtering vastly improves
diversity estimates from illumina amplicon sequencing. Nature Methods
\textbf{10}:57--59.
doi:\href{https://doi.org/10.1038/nmeth.2276}{10.1038/nmeth.2276}.

\leavevmode\hypertarget{ref-Parada2015}{}%
21. \textbf{Parada AE}, \textbf{Needham DM}, \textbf{Fuhrman JA}. 2015.
Every base matters: Assessing small subunit rRNA primers for marine
microbiomes with mock communities, time series and global field samples.
Environmental Microbiology \textbf{18}:1403--1414.
doi:\href{https://doi.org/10.1111/1462-2920.13023}{10.1111/1462-2920.13023}.

\leavevmode\hypertarget{ref-Wang1996}{}%
22. \textbf{Wang GCY}, \textbf{Wang Y}. 1996. The frequency of chimeric
molecules as a consequence of PCR co-amplification of 16S rRNA genes
from different bacterial species. Microbiology \textbf{142}:1107--1114.
doi:\href{https://doi.org/10.1099/13500872-142-5-1107}{10.1099/13500872-142-5-1107}.

\leavevmode\hypertarget{ref-Potapov2017}{}%
23. \textbf{Potapov V}, \textbf{Ong JL}. 2017. Examining sources of
error in PCR by single-molecule sequencing. PLOS ONE
\textbf{12}:e0169774.
doi:\href{https://doi.org/10.1371/journal.pone.0169774}{10.1371/journal.pone.0169774}.

\leavevmode\hypertarget{ref-Kebschull2015}{}%
24. \textbf{Kebschull JM}, \textbf{Zador AM}. 2015. Sources of
PCR-induced distortions in high-throughput sequencing data sets. Nucleic
Acids Research gkv717.
doi:\href{https://doi.org/10.1093/nar/gkv717}{10.1093/nar/gkv717}.

\leavevmode\hypertarget{ref-McInerney2014}{}%
25. \textbf{McInerney P}, \textbf{Adams P}, \textbf{Hadi MZ}. 2014.
Error rate comparison during polymerase chain reaction by DNA
polymerase. Molecular Biology International \textbf{2014}:1--8.
doi:\href{https://doi.org/10.1155/2014/287430}{10.1155/2014/287430}.

\leavevmode\hypertarget{ref-Cline1996}{}%
26. \textbf{Cline J}. 1996. PCR fidelity of pfu DNA polymerase and other
thermostable DNA polymerases. Nucleic Acids Research
\textbf{24}:3546--3551.
doi:\href{https://doi.org/10.1093/nar/24.18.3546}{10.1093/nar/24.18.3546}.

\leavevmode\hypertarget{ref-Acinas2005}{}%
27. \textbf{Acinas SG}, \textbf{Sarma-Rupavtarm R}, \textbf{Klepac-Ceraj
V}, \textbf{Polz MF}. 2005. PCR-induced sequence artifacts and bias:
Insights from comparison of two 16S rRNA clone libraries constructed
from the same sample. Applied and Environmental Microbiology
\textbf{71}:8966--8969.
doi:\href{https://doi.org/10.1128/aem.71.12.8966-8969.2005}{10.1128/aem.71.12.8966-8969.2005}.

\leavevmode\hypertarget{ref-Polz1998}{}%
28. \textbf{Polz MF}, \textbf{Cavanaugh CM}. 1998. Bias in
template-to-product ratios in multitemplate PCR. Applied and
Environmental Microbiology \textbf{64}:3724--3730.

\leavevmode\hypertarget{ref-Brooks2015}{}%
29. \textbf{Brooks JP}, \textbf{David J Edwards}, \textbf{Harwich MD},
\textbf{Rivera MC}, \textbf{Fettweis JM}, \textbf{Serrano MG},
\textbf{Reris RA}, \textbf{Sheth NU}, \textbf{Huang B}, \textbf{Girerd
P}, \textbf{Strauss JF}, \textbf{Jefferson KK}, \textbf{Buck GA}. 2015.
The truth about metagenomics: Quantifying and counteracting bias in 16S
rRNA studies. BMC Microbiology \textbf{15}.
doi:\href{https://doi.org/10.1186/s12866-015-0351-6}{10.1186/s12866-015-0351-6}.

\leavevmode\hypertarget{ref-Suzuki1996}{}%
30. \textbf{Suzuki MT}, \textbf{Giovannoni SJ}. 1996. Bias caused by
template annealing in the amplification of mixtures of 16S rRNA genes by
PCR. Applied and environmental microbiology \textbf{62}:625--630.

\leavevmode\hypertarget{ref-Chandler1997}{}%
31. \textbf{Chandler D}, \textbf{Fredrickson J}, \textbf{Brockman F}.
1997. Effect of pcr template concentration on the composition and
distribution of total community 16S rDNA clone libraries. Molecular
Ecology \textbf{6}:475--482.

\leavevmode\hypertarget{ref-Wagner1994}{}%
32. \textbf{Wagner A}, \textbf{Blackstone N}, \textbf{Cartwright P},
\textbf{Dick M}, \textbf{Misof B}, \textbf{Snow P}, \textbf{Wagner GP},
\textbf{Bartels J}, \textbf{Murtha M}, \textbf{Pendleton J}. 1994.
Surveys of gene families using polymerase chain reaction: PCR selection
and pcr drift. Systematic Biology \textbf{43}:250--261.

\leavevmode\hypertarget{ref-Hansen1998}{}%
33. \textbf{Hansen MC}, \textbf{Tolker-Nielsen T}, \textbf{Givskov M},
\textbf{Molin S}. 1998. Biased 16S rDNA pcr amplification caused by
interference from dna flanking the template region. FEMS Microbiology
Ecology \textbf{26}:141--149.

\leavevmode\hypertarget{ref-Kennedy2014}{}%
34. \textbf{Kennedy K}, \textbf{Hall MW}, \textbf{Lynch MDJ},
\textbf{Moreno-Hagelsieb G}, \textbf{Neufeld JD}. 2014. Evaluating bias
of illumina-based bacterial 16S rRNA gene profiles. Applied and
Environmental Microbiology \textbf{80}:5717--5722.
doi:\href{https://doi.org/10.1128/aem.01451-14}{10.1128/aem.01451-14}.

\leavevmode\hypertarget{ref-Edgar2011}{}%
35. \textbf{Edgar RC}, \textbf{Haas BJ}, \textbf{Clemente JC},
\textbf{Quince C}, \textbf{Knight R}. 2011. UCHIME improves sensitivity
and speed of chimera detection. Bioinformatics \textbf{27}:2194--2200.
doi:\href{https://doi.org/10.1093/bioinformatics/btr381}{10.1093/bioinformatics/btr381}.

\leavevmode\hypertarget{ref-Williams2006}{}%
36. \textbf{Williams R}, \textbf{Peisajovich SG}, \textbf{Miller OJ},
\textbf{Magdassi S}, \textbf{Tawfik DS}, \textbf{Griffiths AD}. 2006.
Amplification of complex gene libraries by emulsion PCR. Nature Methods
\textbf{3}:545--550.
doi:\href{https://doi.org/10.1038/nmeth896}{10.1038/nmeth896}.

\leavevmode\hypertarget{ref-Edgar2017}{}%
37. \textbf{Edgar RC}. 2017. UNBIAS: An attempt to correct abundance
bias in 16S sequencing, with limited success.
doi:\href{https://doi.org/10.1101/124149}{10.1101/124149}.

\leavevmode\hypertarget{ref-Gohl2016}{}%
38. \textbf{Gohl DM}, \textbf{Vangay P}, \textbf{Garbe J},
\textbf{MacLean A}, \textbf{Hauge A}, \textbf{Becker A}, \textbf{Gould
TJ}, \textbf{Clayton JB}, \textbf{Johnson TJ}, \textbf{Hunter R},
\textbf{Knights D}, \textbf{Beckman KB}. 2016. Systematic improvement of
amplicon marker gene methods for increased accuracy in microbiome
studies. Nature Biotechnology \textbf{34}:942--949.
doi:\href{https://doi.org/10.1038/nbt.3601}{10.1038/nbt.3601}.

\leavevmode\hypertarget{ref-Quince2011}{}%
39. \textbf{Quince C}, \textbf{Lanzen A}, \textbf{Davenport RJ},
\textbf{Turnbaugh PJ}. 2011. Removing noise from pyrosequenced
amplicons. BMC Bioinformatics \textbf{12}:38.
doi:\href{https://doi.org/10.1186/1471-2105-12-38}{10.1186/1471-2105-12-38}.

\leavevmode\hypertarget{ref-Sze2016}{}%
40. \textbf{Sze MA}, \textbf{Schloss PD}. 2016. Looking for a signal in
the noise: Revisiting obesity and the microbiome. mBio \textbf{7}.
doi:\href{https://doi.org/10.1128/mbio.01018-16}{10.1128/mbio.01018-16}.

\leavevmode\hypertarget{ref-Sze2018}{}%
41. \textbf{Sze MA}, \textbf{Schloss PD}. 2018. Leveraging existing 16S
rRNA gene surveys to identify reproducible biomarkers in individuals
with colorectal tumors. mBio \textbf{9}.
doi:\href{https://doi.org/10.1128/mbio.00630-18}{10.1128/mbio.00630-18}.

\leavevmode\hypertarget{ref-Multinu2018}{}%
42. \textbf{Multinu F}, \textbf{Harrington SC}, \textbf{Chen J},
\textbf{Jeraldo PR}, \textbf{Johnson S}, \textbf{Chia N},
\textbf{Walther-Antonio MR}. 2018. Systematic bias introduced by genomic
DNA template dilution in 16S rRNA gene-targeted microbiota profiling in
human stool homogenates. mSphere \textbf{3}.
doi:\href{https://doi.org/10.1128/msphere.00560-17}{10.1128/msphere.00560-17}.

\leavevmode\hypertarget{ref-Salter2014}{}%
43. \textbf{Salter SJ}, \textbf{Cox MJ}, \textbf{Turek EM},
\textbf{Calus ST}, \textbf{Cookson WO}, \textbf{Moffatt MF},
\textbf{Turner P}, \textbf{Parkhill J}, \textbf{Loman NJ},
\textbf{Walker AW}. 2014. Reagent and laboratory contamination can
critically impact sequence-based microbiome analyses. BMC Biology
\textbf{12}.
doi:\href{https://doi.org/10.1186/s12915-014-0087-z}{10.1186/s12915-014-0087-z}.

\leavevmode\hypertarget{ref-Callahan2016}{}%
44. \textbf{Callahan BJ}, \textbf{McMurdie PJ}, \textbf{Rosen MJ},
\textbf{Han AW}, \textbf{Johnson AJA}, \textbf{Holmes SP}. 2016. DADA2:
High-resolution sample inference from illumina amplicon data. Nature
Methods \textbf{13}:581--583.
doi:\href{https://doi.org/10.1038/nmeth.3869}{10.1038/nmeth.3869}.

\leavevmode\hypertarget{ref-Edgar2016}{}%
45. \textbf{Edgar RC}. 2016. UNOISE2: Improved error-correction for
illumina 16S and ITS amplicon sequencing.
doi:\href{https://doi.org/10.1101/081257}{10.1101/081257}.

\leavevmode\hypertarget{ref-Amir2017a}{}%
46. \textbf{Amir A}, \textbf{McDonald D}, \textbf{Navas-Molina JA},
\textbf{Kopylova E}, \textbf{Morton JT}, \textbf{Xu ZZ},
\textbf{Kightley EP}, \textbf{Thompson LR}, \textbf{Hyde ER},
\textbf{Gonzalez A}, \textbf{Knight R}. 2017. Deblur rapidly resolves
single-nucleotide community sequence patterns. mSystems \textbf{2}.
doi:\href{https://doi.org/10.1128/msystems.00191-16}{10.1128/msystems.00191-16}.

\leavevmode\hypertarget{ref-Schloss2018}{}%
47. \textbf{Schloss PD}. 2018. Identifying and overcoming threats to
reproducibility, replicability, robustness, and generalizability in
microbiome research. mBio \textbf{9}.
doi:\href{https://doi.org/10.1128/mbio.00525-18}{10.1128/mbio.00525-18}.

\leavevmode\hypertarget{ref-Seekatz2016}{}%
48. \textbf{Seekatz AM}, \textbf{Rao K}, \textbf{Santhosh K},
\textbf{Young VB}. 2016. Dynamics of the fecal microbiome in patients
with recurrent and nonrecurrent clostridium difficile infection. Genome
Medicine \textbf{8}.
doi:\href{https://doi.org/10.1186/s13073-016-0298-8}{10.1186/s13073-016-0298-8}.

\leavevmode\hypertarget{ref-Schloss2009}{}%
49. \textbf{Schloss PD}, \textbf{Westcott SL}, \textbf{Ryabin T},
\textbf{Hall JR}, \textbf{Hartmann M}, \textbf{Hollister EB},
\textbf{Lesniewski RA}, \textbf{Oakley BB}, \textbf{Parks DH},
\textbf{Robinson CJ}, \textbf{Sahl JW}, \textbf{Stres B},
\textbf{Thallinger GG}, \textbf{Horn DJV}, \textbf{Weber CF}. 2009.
Introducing mothur: Open-source, platform-independent,
community-supported software for describing and comparing microbial
communities. Applied and Environmental Microbiology
\textbf{75}:7537--7541.
doi:\href{https://doi.org/10.1128/aem.01541-09}{10.1128/aem.01541-09}.

\leavevmode\hypertarget{ref-Rognes2016}{}%
50. \textbf{Rognes T}, \textbf{Flouri T}, \textbf{Nichols B},
\textbf{Quince C}, \textbf{Mahé F}. 2016. VSEARCH: A versatile open
source tool for metagenomics. PeerJ \textbf{4}:e2584.
doi:\href{https://doi.org/10.7717/peerj.2584}{10.7717/peerj.2584}.

\leavevmode\hypertarget{ref-Westcott2017}{}%
51. \textbf{Westcott SL}, \textbf{Schloss PD}. 2017. OptiClust, an
improved method for assigning amplicon-based sequence data to
operational taxonomic units. mSphere \textbf{2}:e00073--17.
doi:\href{https://doi.org/10.1128/mspheredirect.00073-17}{10.1128/mspheredirect.00073-17}.

\leavevmode\hypertarget{ref-r_citation_2018}{}%
52. \textbf{R Core Team}. 2018. R: A language and environment for
statistical computing. R Foundation for Statistical Computing, Vienna,
Austria.

\newpage

\textbf{Figure 1. The error rate of assembled mock community sequence
reads increases with the number of rounds of PCR; however, much of this
error was eliminated by denoising and followed the relative error rates
provided by the manufacturers.} Each line represents the mean of four
replicates.

\textbf{Figure 2. The fraction of all denoised sequences that were
identified as being chimeric increases with the number of rounds of PCR
used and varied between polymerases.} (A) Sequencing of a mock community
allowed us to identify the total fraction of sequences that were
chimeric as well as the specificity and sensitivity of UCHIME to detect
those chimeras. Each line represents the mean of four replicates. (B)
Sequencing of four human stool samples after using one of five different
polymerases again demonstrated increased rate of chimera formation with
increasing number of rounds of PCR and variation across polymerases.

\textbf{Figure 3. The relative abundances of mock community sequence
reads mapped to reference sequences differed subtly from the expected
relative abundances as determined by shotgun metagenomic sequencing.}
Bias did not increase with number of rounds of PCR or vary by polymerase
or the guanine and cytosine content of the fragment. The expected
relative abundance of each organism is indicated by the horizontal gray
line. The percentage of bases that were guanines or cytosines within the
V4 region of the 16S rRNA genes in each organism is indicated by the
number in the lower left corner of each panel. Each line represents the
mean of four replicates.

\textbf{Figure 4. Despite evidence of subtle PCR bias at the genome
level, there was significant evidence of bias using community-wide
metrics that grew with the number of rounds of PCR when using a mock
community.} (A) With the exception of the KAPA polymerase data, the
richness and Shannon diversity values increased with number of rounds of
PCR and the inclusion of residual sequencing errors and chimeras. The
horizontal black line indicates the expected richness and diversity. (B)
Relative to the mock community sampled after 25 rounds of PCR, the
distance to the communities sampled after 30 and 35 rounds of PCR
increased for all polymerases. (C) The variation between samples
demonstrated a significant change in the community driven by the number
of rounds of PCR and the polymerase used. The ellipses represent
bivariate normally distributed 95\% confidence intervals. The data in A
and B represents the mean of four replicates.

\textbf{Figure 5. Sequencing of human stool samples indicated clear
increase in bias with number of rounds of PCR, however, the bias
appeared to be consistent within each sample.} (A) With the exception of
data collected using the KAPA polymerase, the richness and Shannon
diversity values increased with number of rounds of PCR. (B) Relative to
the stool communities sampled after 25 rounds of PCR, the distance to
the stool communities sampled after 30 and 35 rounds of PCR was
inconsistent and there was little difference in variation for data
collected using the KAPA polymerase. (C) The variation between stool
samples was larger than the amount of variation introduced by varying
the number of rounds of PCR or polymerase. The ellipses represent
bivariate normally distributed 95\% confidence intervals. Results for
some samples at 20 cycles are not presented because it was not possible
to obtain a sufficient number of reads for those polymerases.

\textbf{Figure 6. The average distance between replicates of sequencing
the same mock community or between the human stool samples (i.e.~drift)
did not vary by number of rounds of PCR or by polymerase.} .

\textbf{Figure S1: With the exception of the sequence data generated
using the KAPA polymerase, the ratio of the two \emph{Salmonella
enterica} V4 sequences from the mock community was lower than the
expected ratio of 6:1.} The dominant and rare \emph{S. enterica} V4
sequences differed by a single base. The horizontal gray line indicates
the expected 6:1 ratio. Each line represents the mean of four
replicates.


\end{document}
